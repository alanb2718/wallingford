\documentclass{article}
\setlength{\topmargin}{-0.25 in}
\setlength{\textwidth}{6.5in}
\setlength{\textheight}{8.5 in}
\setlength{\oddsidemargin}{0in}
\setlength{\evensidemargin}{0in}
\setlength{\parindent}{0.0in}
\setlength{\parskip}{7pt plus 1pt}

\usepackage{url}

\sloppy

\begin{document}

\title{Toward a Constraint Reactive Programming Language}
\author{Alan Borning}
\date{\today}

\maketitle

\section{Introduction}

This note outlines a constraint reactive programming language that
extends Wallingford with temporal constraints and reactivity.  Many of
the examples and features are inspired by Fran \cite{elliot-icfp-1997},
although, as an imperative language, in many ways it is not at all
like Fran.  The language uses a hybrid automata approach, loosely
modeled on that described in \cite{alur-lncs-1993}.

The examples in this note don't run yet --- they are just to explore
the language.

The basic intuition is that we are modeling or simulating a continuous
system (or better, a set of interacting continuous systems) on a
discrete digital computer.\footnote{Eventually we might extend this
  model to include actual physical systems with sensors and
  actuators.}  An application consists of one or more interacting
processes.  Each process has a continuously varying clock, and also a
program counter, which ranges over a set of control locations.  The
state of each process can change continuously with time, as specified
by the currently active set of constraints.  The processes can also be
reactive --- when some event occurs, or a condition changes its truth
value, the system can transition to a new state.  This might involve
adding or removing constraints, and also assigning to variables.

The current time in any process can be referenced using the distinguished
variable \verb|time|.  In the examples that follow I'll assume lexical
scoping and a default process, so that \verb|time| can be simply
referenced.  \verb|time| is guaranteed to increase monotonically, but not
at a particular rate relative to wall clock time.  For decent animations,
though, it should increase at a fixed rate, at least approximately.  Its
type is just real.  (At one point in the design there was a separate type
Continuous, but this is now dropped.)  There is also a predefined process
named \verb|system|, whose clock is the hardware clock, and that also can
be queried for the state of input devices.  Another predefined process is
\verb|display|, for output.  The display process takes care of sampling at
an appropriate rate to produce smooth animations and interactions if
possible.

In Fran, the abstract domain \emph{Time} has some non-standard features,
since we want to be able to know that a given time is at least some
value, even though we don't know yet exactly what that final value will be.
So in Fran, we can have a specific time (for example 100), and also a time
that is at least 42, written $_{\geq} 42$.  These issues have been ignored
for the moment in Wallingford, but we need to return to this and decide
whether to adopt a seminar semantics.

A simpler starting point for the design may be to just have one process
initially, and later figure out multiple interacting processes.  For now,
though, this note uses multiple interacting processes.

An event is a boolean-valued variable that denotes that something
happened at a particular instant.  (Again, at one point in the design
there was a separate type Event, but this is also dropped in favor of
the type just being boolean.)  An event models some discrete
occurrence --- there will be only a finite number of instants in which
the value of the event variable is true in any given time interval.
(If we define another variable V whose value is 0 or 1 depending on
whether the event variable is false or true, the integral of V over
any finite time interval will be 0.  Not sure about infinite
intervals, but it probably doesn't matter.  Hopefully someone who is
better at theory than me can clean this up.)

An important concept is that of \emph{observing} or \emph{sampling}.
Semantically, time varies continuously, but we can only observe or
sample the state of a process at discrete times.  By defining appropriate
events in the process being observed (using `when' --- see below), we can
avoid missing important states, such as things that might be true just for
an instant.

The language includes the following new macros, in addition to the
functions and macros in the current Wallingford implementation.  (The
existing ones are \verb|always|, \verb|always*|, \verb|stay|,
\verb|wally-clear|, \verb|wally-solve|, and \verb|wally-solve+|.)

\begin{description}
    
\item[when] takes an event-valued expression and evaluates some
  statements at the instant that event occurs.  These can include
  assignments and assertions of new constraints.  (Note that this is not
  the same as the built-in \verb|when| control structure in Racket ---
  for a prototype implementation, we could call this \verb|wallywhen|.)

\item[while] asserts a set of constraints that should hold as long as
  the condition has the value true.

\item[integral] \verb|(integral f t0)| is the definite integral
  $\int_{t_0}^t f \, dt$, where $f$ is some expression, $t_0$ is a start
  time and $t$ is the current time.

\item[derivative] \verb|(derivative e)| evaluates to the derivative of
  \verb|e| with respect to the current time.

\item[detect] takes a boolean-valued expression and returns an event
  that is true at the instant the value of the expression changes to
  true.

\end{description}

The default duration for a \verb|while| or \verb|when| macro is always: it
should hold from that point onward.  These macros also allow for an
\verb|until| clause with an event-valued expression, which specifies that
the \verb|while| or \verb|when| should be deactivated if the event occurs.
Ordinary constraints can still have the {\tt always} and {\tt once}
annotations as in the current Wallingford, and ordinary constraints can
also use the \verb|until| annotation instead.  (However, \verb|once|
doesn't make sense for \verb|while| or \verb|when| --- an equivalent
\verb|if| statement would be simpler and clearer and give the same result.)

We'll use objects and messages for processes.  So we can query for the
system time or state of the mouse as follows:
\begin{verbatim}
  (send system time)
  (send system left-button-down) 
  (send system left-button-down-event)
\end{verbatim}

The \verb|left-button-down| message returns true if the button is down at
that time.  The \verb|left-button-down-event| returns true just at the
instant the button is pressed; otherwise it is false.

\subsection{Relation to Fran}

Fran is embedded in Haskell, and as such is a functional language.  In
Fran, behaviors are functions of time.  The time parameter is implicit ---
the system takes care of supplying it to get the value of the behavior
(i.e.\ function) when it is needed.  There is no state; we can evaluate a
behavior on many different times without any interference among them.  In
Wallingford, on the other hand, the value of an expression can vary with
time.

\section{Examples and Discussion}

This section includes a discussion of some of the new features, along with
a series of examples, mostly adapted from the Fran paper
\cite{elliot-icfp-1997}.

\subsection{Basic Graphical Objects}
\label{sec:basic-graphics}

Regarding how graphical objects are constructed, Fran starts with a
constant unit circle as well as other geometric constants, and computes new
values by transforming these existing values.  (Since it's purely
functional, we can only make new values and not alter existing ones.)
While we could do that in Wallingford, we instead take a more stateful
approach, using either a struct or a class \verb|Circle|.  We constrain its
size relative to \verb|time|, so that it smoothly grows and shrinks.

%% \begin{verbatim}
%% c := Circle.new();
%% c.center := point(10,20);
%% /* this causes the circle to keep growing and shrinking */
%% always c.radius = 100 + 100*sin(system.time)
%% \end{verbatim}

First, here is a version that uses a \verb|circle| struct.  There is a
function \verb|make-circle| that returns an instance of this struct, with
the center a symbolic point and the radius a symbolic number.

\begin{verbatim}
(struct point (x y) #:transparent)
(struct circle (center radius color) #:transparent)

(define (make-point)
  (define-symbolic* x y number?)
  (point x y))

; This version takes an initial center and radius for the circle (the
; center and radius themselves are symbolic quantities).  The color is just
; fixed however.
(define (make-circle initial-center initial-radius [color 'black])
  (define c (make-point))
  (define-symbolic* r number?)
  (assert (equal? c initial-center))
  (assert (equal? r initial-radius))
  (wally-solve)
  (circle c r color))

(define c (make-circle (point 100 200) 0))
(always (equal? (circle-radius c) (+ 100 (* 100 (sin time)))))
(wally-solve)
\end{verbatim}

We've made the constraint on the radius an \verb|always| constraint;
as a result the radius will continue to track time indefinitely.  (In
the implementation, if nobody can possibly see or access the circle
any more, it could be garbage collected.)  An alternative would be to
use an \verb|until| clause to specify a condition for removing that
constraint.

There is an external \verb|display| process (not shown) that samples
the circle's state to produce a smooth animation of it growing and
shrinking.

Here is an alternate version that uses a class.

\begin{verbatim}
(define c (new circle% [center (new point% 100 200)] [radius 100]))
(always (equal? (send c radius) (+ 100 (* 100 (sin time)))))
(wally-solve)
\end{verbatim}

From now on, we will just use structs for user-defined compound data,
but these could just as easily be classes and real objects, assuming
there isn't an issue with classes and Rosette.  (Structs are lifted in
Rosette, but the object-oriented facilities in Racket are not.)

\subsection{Using Events}

%% Babelsberg version:
%% \begin{verbatim}
%% always at5 = (system.time^2 = 5)
%% \end{verbatim}

An event in Fran represents the fact that something happened at a
given time.  In the formal semantics, the semantic function {\bf occ}
maps an event to a pair $(\emph{Time},\alpha)$.  It is non-strict.
I'm still confused about this aspect of Fran, so this may not be
entirely correct.  Pressing on despite this confusion, in Wallingford
events are represented as boolean-valued variables that
instantaneously have the value true at given times; at all other times
they have the value false.

The Fran paper includes a number of examples of events.  One is
\verb|(predicate (time^2) == 5) t0|.  In Wallingford we could
define an analogous variable \verb|at5| as follows:

\begin{verbatim}
(define-symbolic at5 boolean?)
(always (equal? at5 (equal? (expt time 2) 5)))
\end{verbatim}

\verb|at5| is a boolean-valued variable that represents an event.  It
will be true for the instant when the system time squared is exactly
5, and otherwise false.  (We'll deal with roundoff errors and such as
part of the implementation of interval analysis --- for now let's view
numeric variables as real-valued, not float-valued.)

%% Babelsberg version:
%% \begin{verbatim}
%% e1 = first(system.leftButtonPressedEvent, system.time, t0)
%% \end{verbatim}

%% \begin{verbatim}
%% def flip(c)
%%   if c=red then black else red;
%% end;

%% c := red;  /* initial value for c */
%% when system.leftButtonPressedEvent
%%     c := flip(c);
%% end;
%% \end{verbatim}

Here is an example of using events with the \verb|when| macro, derived from
a Fran example in which we cycle a color between red and black each time
the left mouse button is pressed.

\begin{verbatim}
(define c 'red)
(define (flip c) (if (eq? c 'red) 'black 'red))
(when (send system left-button-down-event)
  (set! c (flip c)))
\end{verbatim}

The \verb|when| macro says that at the instant the event occurs, the
variable \verb|c| is set to red if it is currently black, and to black if
it is currently red.  This is making use of the hybrid automata approach
described in the introduction.  The clock advances continuously.  At the
instant the left button is pressed, the statements in the body of the
\verb|when| are executed, and so \verb|c| changes color.  (This happens
instantaneously as far as the continuous real-time clock is concerned.)
Note that \verb|set!| will need to be modified to evaluate the
right-hand-side with respect to the current solution to the constraints and
also to account for any constraints on the variable being set.  The
\verb|wally-set!| macro defined in \verb|babelsberg/babelsberg.rkt| does
what is required here.  Alternatively, we could replace this expression
with the following, which accomplishes the same thing without needing to
redefine \verb|set!|.
\begin{verbatim}
(assert (eq? c (flip (evaluate c))))
\end{verbatim}

%% \begin{verbatim}
%% class Wiggler
%%   instvars value;
%%   def init(lo, hi)
%%     -- set up constraints
%%     always wiggle = sin(pi*system.time) 
%%     always self.value = (hi-lo)? * (wiggle+1)/2
%%     /* We left lo, hi, and wiggle as just local variables rather than
%%        making them be instance variables, although if we wanted access
%%        to them later we could make them be instance vars.  wiggle is
%%        read-only since system.time is read-only; and the ? on the
%%        expression (hi-lo) means that only self.value can change to
%%        satisfy the constraint. */
%%   end;
%% \end{verbatim}

Following the Fran example of a wiggler behavior, the following function
makes an instance of a wiggler \verb|struct| whose value oscillates
smoothly between \verb|lo| and \verb|hi|.
\begin{verbatim}
(struct wiggler (value))
(define (make-wiggler lo hi)
  (define-symbolic* value wiggle number?)
  (always (equal? wiggle (sin (* pi time))))
  (always (equal? value (* (- hi lo) (+ wiggle 1) 0.5)))
  (wiggler value))
\end{verbatim}

%% \begin{verbatim}
%% pBall := Circle.new;
%% pBall.center := point(10,20);
%% pBall.color := red;
%% /* this causes the circle to keep growing and shrinking */
%% wig := Wiggler.new(0.5,1);
%% always pBall.radius = wig.value;

%% /* rBall that tracks the changes in the size of  pBall */
%% rBall := Circle.new;
%% rBall.center := point(10,50);
%% rBall.color := red;
%% /* make rBall always be 1/10 the size of pBall
%% always rBall.radius = pBall.radius * 0.1;
%% \end{verbatim}

Using \verb|make-wiggler|, we can define a red pulsing ball \verb|pBall|,
and another ball \verb|rBall| that is 1/10 its size.  This uses the
\verb|circle| struct defined in Section \ref{sec:basic-graphics}.

\begin{verbatim}
(define pBall (make-circle (point 10 20) 10 'red))
(define rBall (make-circle (point 10 50) 1 'red))
(define wig (make-wiggler 0.5 1.0))
(always (equal? (circle-radius pBall) (wiggler-value wig)))
(always (equal? (circle-radius rBall) (* 0.1 (circle-radius pBall))))
\end{verbatim}

%% \begin{verbatim}
%% always im.center = system.mousePosition;
%% \end{verbatim}

This bit of code makes an image follow the mouse (indefinitely):

\begin{verbatim}
(struct image (center bits))
(define im (image (point 100 100) (get-jpg "picture.bits")))
(always (equal? (image-center im) (send system mouse-pos)))
\end{verbatim}


%% \begin{verbatim}
%% while system.leftButtonPressed
%%   im.center = system.mousePosition;
%% end;
%% \end{verbatim}

More likely, we want the image to follow the mouse only when for
example a button is down:

\begin{verbatim}
(while (send system left-button-down)
  (equal? (image-center im) (send system mouse-pos)))
\end{verbatim}

\subsection{Integration and Differentiation}
\label{sec:integration}

Fran has integration with respect to time.  We should too.
\verb|(integral f t0)| is the definite integral $\int_{t_0}^t f \, dt$,
where $f$ is some expression, $t_0$ is a start time and $t$ is the current
time.  There is also a \verb|derivative| macro, where \verb|(derivative e)|
evaluates to the derivative of \verb|e| with respect to the current
time.\footnote{Integration in Wallingford is not very general: it is
  restricted to being with respect to time, and the upper limit is
  implictly the current time.  These seem like reasonable choices if the
  function being integrated is represented simply as a time-varying
  expression.  (If we wanted a more general kind of integral macro, we
  would want $f$ to be a function, for example.)  Also, Fran doesn't have
  derivative \ldots\ maybe omit it for Wallingford as well?}


As a first example, here is a ball with a position and velocity, and
acceleration due to gravity.  (For simplicity, we ignore the ball's radius;
it wouldn't be hard to include but just adds clutter.)  Initially here it
is with no possibility of collisions.

%% \begin{verbatim}
%% class NoCollisionBall
%%   def init
%%     initialPos := self.position;
%%     initialVel := 0;
%%     always self.pos = initialPos? + integral(self.vel,self.time);
%%     always self.vel = initialVel? + integral(g,time);
%%     /* g is the acceleration due to gravity */    
%%   end;
%% \end{verbatim}

\begin{verbatim}
(struct ball (position velocity))
(define (make-ball p0)
  ; p0 is the initial position of the ball, and v0 is its initial velocity
  ; p and v are the current position and velocity respectively
  (define t0 time)
  (define v0 0)    ; not really necessary to make this a variable, but seems tidier
  (define-symbolic* p v number?)
  (always (equal? p (+ p0 (integral v t0))))
  ; g is the acceleration due to gravity
  (always (equal? v (+ v0 (integral g t0))))
  (ball p v))
\end{verbatim}

Here is the class \verb|Ball| again, this time inside a box.  It
bounces off the walls, reducing its velocity in that case by the
coefficient of restitution.  (The Fran paper calls this `reciprocity'
but COR seems to be more standard.)

\begin{verbatim}
(define (make-ball startpos box)
  ; box is the bounding box that it bounces around in
  ; p0 is originally the initial position of the ball, and after
  ; the first collision is its position at the last collision
  (define p0 startpos)
  (define t0 time)
  (define v0 0)
  (define cor 0.8)   ; coefficient of restitution
  (define-symbolic* p v number?)
  (while (not (collision ball box))
    (equal? p (+ p0 (integral v t0)))
    (equal? v (+ v0 (integral g t0))))
  (when (collision ball box)
    (set! v (compute-bounce-velocity ball box cor))
    (set! v0 v)
    (set! p0 p))
  (ball p v))
    
(define (collision ball box)
  ; return true if the box doesn't contain the ball
  ; (or we could instead check whether the ball is exactly on a side of the box)
  [code not shown])

; reflect the velocity depending on which wall we hit, reducing it
; by the coefficient of restitution 
(define (compute-bounce-velocity ball box cor)
  (if (or (equal? (rectangle-top box) (point-y (ball-position ball)))
          (equal? (rectangle-bottom box) (point-y (ball-position ball))))
      (point (* (point-x (ball-velocity ball) cor))
             (* -1 (point-y (ball-velocity ball) cor)))
      (point (* -1 (point-x (ball-velocity ball) cor))
             (* (point-y (ball-velocity ball) cor)))))
\end{verbatim}

%% \begin{verbatim}
%% class Ball
%%   def init
%%     self.box := .... [the bounding box that it bounces around in];
%%     /* the coefficient of restitution */
%%     self.cor = 0.8;
%%     /* startPos is originally the initial position of the ball, and
%%        following the first collision is its position at the last collision */
%%     startPos := self.position;
%%     /* similarly for startVel */
%%     startVel := 0;
%%     while not(collision)
%%       self.pos = startPos? + integral(self.vel,self.time);
%%       self.vel = startVel? + integral(g,time);
%%       /* g is the acceleration due to gravity */    
%%     end;
%%     when collision
%%       self.vel := self.compute_bounce_velocity();
%%       startVel := self.vel;
%%       /* note that we don't need to change self.pos */
%%       startPos := self.pos  /* save the current position in startPos */
%%     end;
%%   end;

%%   def collision
%%     return not(self.box.contains(self.pos));
%%     /* we could instead check whether self.pos is exactly on a side of the box */
%%   end;

%%   /* reflect the velocity depending on which wall we hit, reducing it
%%      by the coefficient of restitution */
%%   def compute_bounce_velocity
%%     if (self.pos.y = self.box.top) | self.pos.y = self.box.bottom
%%       return self.vel * vector(1,-1) * self.cor;
%%     else
%%       return self.vel * vector(-1,1) * self.cor;
%%     end;
%%   end;

%% end;
%% \end{verbatim}

\subsection{Constraining the Clock}

In the examples so far we've just used the \verb|time| of the current
process without putting any constraints on it.  We could do so however.
For example, following Fran, we could make the clock in the current process
lag 2 seconds behind the system clock:

\begin{verbatim}
(always (equal? time (+ 2 (send system time))))
\end{verbatim}

Or slow it down by a factor of 2:

\begin{verbatim}
(always (equal? time (* 0.5 (send system time))))
\end{verbatim}

\section{Implementation}

This section is just some rough notes at this point!

Try to do this in Rosette, building on the current Wallingford
implementation.  It's liable to be exceedingly slow --- the initial
plan would be to not worry about this, but get the semantics right.
Later, consider how to implement it more efficiently.

Interval analysis, sampling, and \verb|when| and \verb|while| macros
would be handled as follows.  The whole thing is driven by the
\verb|display| process.  If the \verb|display| process wants to take a
time step of $\delta$, we would verify that there are no times between
the current time $t$ and $t+\delta$ such that the condition on any
\verb|when| macros is true, and that the truth value of the condition
on any \verb|while| macro remains unchanged.  If so, we can safely
advance time by $\delta$.  If not, try advancing it by $\delta / 2$;
continue until the conditions are met or the $\delta$ is so small that
there is no visible change from applying it.  (Does this work??  Or
maybe we could find an actual value for $\delta$ such that a
\verb|when| condition becomes true or a \verb|while| condition changes
from true to false or vice versa.  Need to figure this out.)

\section{Alternatives to the Hybrid Automata Approach}

The hybrid automata approach has the important advantage that it has
been developed formally in prior work, hopefully allowing us to build
on that formalism.  However, the two kinds of time are a bit
unsatisfying.  What if we only wanted one kind?

\subsection{Continuous Time Only}

In this version we have only continuous time, dropping the imperative
programming constructs altogether, including the program counter and
discrete time.  Instead, the program would consist only of constraints
(including temporal constraints) --- but no sequential code.  There
would be no assignment statements.  In place of assignment statements
in initialization code, the programmer would use \verb|once|
constraints.  In place of assignment statements in \verb|when|
statements, the language would include a \verb|prev| macro, with
\verb|(prev expr)| evaluating to the state of \verb|expr| at time 
$t - \delta$ for some $\delta$, with $\delta$ chosen to be small
enough so that nothing in the program uses any times closer to $t$.
The \verb|prev| expression would be be read-only, that is, the
constraint solver can only find solutions that reference them but that
don't affect them.  (See \cite{borning-lisp-symbolic-computation-1992}
for a formal definition of ``read-only.'')

Here is the flip example using \verb|prev|.

\begin{verbatim}
(define c 'red)
(define (flip c) (if (eq? c 'red) 'black 'red))
(when (send system left-button-down-event)
  (assert (eq? c (flip (prev c)))))
\end{verbatim}

(A variant would be to provide a \verb|next| macro rather than or in
addition to \verb|prev| --- in this case \verb|expr| in
\verb|(next expr)| would be \emph{write-only} for the constraint
solver.)

I hypothesize that this version has the same expressive power as the
one based on hybrid automata.  This could be demonstrated by showing how to
transform one version into the other.  Sketch of how to do this:

\begin{itemize}

\item One direction: whenever there is an occurrence of \verb|(prev expr)|
  in some assert, insert an assignment \verb|(set! temp expr)| before the
  assert (using the \verb|wally-set!| version of \verb|set!|), and replace
  \verb|(prev expr)| in the assert with \verb|temp|.

\item Other direction: replace assignments \verb|(set! var expr)| with
  \verb|(assert (equal? var (prev expr)))|.  We need to deal with the fact
  that this yields simultaneous assignments, rather than sequential ones,
  but it seems likely that there is a way to do this by introducing
  additional variables.

\end{itemize}

\subsection{Discrete Time Only}

This hasn't been worked out yet.  This version would very likely have the
usual disadvantages of discrete time simulations, for example, having to
worry about sampling and missing events such as the bouncing ball hitting
the floor.  We might need to fall back to using just discrete time, but I
first want to try continuous time.

\bibliographystyle{plain}
\bibliography{constraints}

\end{document}
