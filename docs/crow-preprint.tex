% position paper for CROW workshop 2016

\documentclass[reprint]{sigplanconf}
% numbers       To obtain numeric citation style instead of author/year.

\usepackage{amsmath,url}

\begin{document}
\toappear{\normalsize To appear, Proceedings of the Constrained and
  Reactive Objects Workshop (CROW'16), March 15, 2016, M\'{a}laga, Spain}

\sloppy

\conferenceinfo{CROW'16}{March 15, 2016, M\'{a}laga, Spain}
\copyrightyear{2016}
\proceedings{Proceedings of the Constrained and Reactive Objects Workshop}

\special{papersize=8.5in,11in}
\setlength{\pdfpageheight}{\paperheight}
\setlength{\pdfpagewidth}{\paperwidth}


\title{Wallingford: Toward a Constraint \\  Reactive Programming Language}

\authorinfo{Alan Borning}
{University of Washington, USA and Communications Design Group}
{borning@cs.washington.edu}

\maketitle

\begin{abstract}
Wallingford is an experimental constraint reactive programming language
that supports objects with state, persistent constraints on those objects,
and reactive constraints that enable the system to respond to events by
declaring constraints that become active when an event occurs, or while a
condition holds.  It is implemented in Rosette, a solver-aided programming
language and environment for constructing DSLs in turn implemented in
Racket.  This short paper is a snapshot of work-in-progress on the system.

\end{abstract}

\category{D.3.3}{Programming Languages}
         {Language Constructs and Features}
         [Constraints]

\keywords{Object constraint languages, reactive programming, Rosette}

section{Introduction}

Wallingford is an experimental constraint reactive programming
language, which integrates constraints with reactive programming, in a
way that cleanly accommodates state and change, and that at the same
time favors being declarative rather than imperative when feasible.
The initial target applications are interactive graphical ones that
use object-oriented features, and constraints to specify relations
among the graphical elements and also how the application reacts to
inputs.  The language draws on previous work on Babelsberg
\cite{FelgentreffJOT14,felgentreff-oopsla-2015}, a family of
object constraint languages, and is implemented in Rosette
\cite{torlak-onward-2013,torlak-pldi-2014}, a
solver-aided language that extends Racket (a Scheme dialect).

This position paper is a snapshot of the work to date.  Section
\ref{sec:related-work} discusses three key influences on the design of
Wallingford, namely Babelsberg, Rosette, and the Fran
functional reactive programming system
\cite{elliot-icfp-1997}.  Section \ref{sec:core-wallingford} describes a small
set of macros and methods that extend Rosette for use in representing
objects with state and persistent constraints, while Section
\ref{sec:reactive} discusses adding reactive constraints.

A prototype implementation of the core Wallingford system described in
Section \ref{sec:core-wallingford} is complete and reasonable stable.
The additions to support reactive constraints (Section
\ref{sec:reactive}) are being implemented, with some but not all being
currently operational.  The current system supports some simple
interactive demonstrations, but performance is explicitly not a goal
at this time.  Rather, the goal is to work out a clean and expressive
language; at the same time, the language should not include features
that would make it impossible to develop much faster implementations
later.  All code is available in the repository
\verb|https://github.com/cdglabs/wallingford.git|

\section{Related Work}
\label{sec:related-work}

Arguably the most common approach to integrating constraints with a
programming language is to extend logic programming, for example as is
done in the constraint logic programming scheme
\cite{jaffar-popl-1987}, or by using integrated constraint libraries
as in SWI Prolog.  There are a number of significant advantages to
this approach, including a generally clean semantics and programming
model.  However, it does not provide state or reactivity, which are
two goals for Wallingford.  Instead, Wallingford draws more directly
on languages and systems that do provide state and reactivity.  This
section describes three: the Babelsberg family of object constraint
languages, the Rosette solver-aided language, and the functional
reactive animation system Fran.

\subsection{Babelsberg}
\label{sec:babelsberg}

Babelsberg is a family of
object constraint languages that allows constraints to be integrated with
an existing object-oriented language.  Current
implementations are Babelsberg/Ruby, Babelsberg/Javascript, and
Babelsberg/Squeak (Squeak being a Smalltalk dialect).

Babelsberg enables the programmer to write constraints on objects that
include OO constructs such as message sends and that respect object
encapsulation.  For example, this constraint specifies that the center
of a rectangle \verb|r| remain at a fixed position as its other
attributes are updated.\footnote{The examples in this section are
  written in pseudo-code, following the examples in 
  \cite{felgentreff-step-by-step-tr}.}

\begin{verbatim}
always r.center() = Point(10,20)
\end{verbatim}

It doesn't matter whether \verb|center| is stored as a field of the
rectangle or computed via a method; in either case, Babelsberg will
evaluate the constraint expression in constraint construction mode to
rewrite it into a set of primitive constraints that can then be turned over
to the solver.

A key aspect of Babelsberg is the handling of state, assignment, and object
identity \cite{felgentreff-oopsla-2015}.  Babelsberg makes a distinction
between instances of primitive types and of value classes, for which object
identity is not significant and that don't need to be stored on the heap,
and instances of classes for which identity is significant and that are
stored on the heap.  Here is a simple example using just primitive types.

\newpage

\begin{verbatim}
x := 3;
y := x;
always y=10;
\end{verbatim}

At the end of this fragment we have \verb|x=3| and \verb|y=10|.  \verb|x|
and \verb|y| hold instances of a primitive type, and so even though we
assign \verb|x| to \verb|y|, they are not aliased, and the \verb|always|
constraint only affects \verb|y|.  In addition to explicit constraints
(such as \verb|y=10| above), there are implicit soft stay constraints to
give stability in the absence of other stronger constraints --- so after
the initial assignment to \verb|x|, there is a low-priority stay constraint
that it retain the value 3.  There is also a low-priority stay
constraint on \verb|y| after the assignment \verb|y:=x|, but this stay is
overridden by the required constraint \verb|y=10|.

On the other hand, for ordinary objects that aren't primitive types or
instances of value classes, object identity \emph{is} significant.  Aliases
can be either implicitly created using an assignment statement, or by using
explicit identity constraints.  Reference \cite{felgentreff-oopsla-2015}
discusses Babelsberg's approach to taming the power of the constraint
solver so that it can't on its own create new objects with object identity
to satisfy constraints, add fields or methods to objects, or the like.

Semantically, all the constraints are solved after each statement.  In
implementations, however, Babelsberg uses various techniques to enable
reasonable performance, such as keeping track of dependencies so that only
constraints involving changed objects need be re-solved, and ``edit''
constraints to take advantage of incremental solvers.

\subsection{Rosette}
\label{sec:rosette}

Rosette is a solver-aided 
language that extends Racket \cite{racket} with
symbolic data types and solver operations.  The motivation for using
Rosette as an implementation platform for Wallingford are two-fold.  First,
Rosette includes support for inferring constraints and programs from
examples, and this capability will very likely be useful in the future for
automatically inferring graphical constraints from example layouts.
Second, it is a thought-provoking system, and I simply have
an intuition that interesting results will flow from trying to meld it with
the Babelsberg approach to objects and constraints (for example, by
using Rosette's symbolic variables).
The Rosette system supports a variety of constraint solvers --- the
Wallingford implementation uses Z3 \cite{demoura-z3-2008}.

As a simple Rosette example, the following statements declare two
symbolic variables and constraints on them, and then find a solution
that satisfies these constraints.  (The Rosette \verb|solve| macro
takes an expression that can assert additional constraints; since we
aren't using that capability here we just use \verb|#t|.)

\begin{verbatim}
  (define-symbolic x y number?)
  (assert (equal? y (+ x 2)))
  (assert (equal? y 10))
  (solve #t)
\end{verbatim}

The previous example of a constraint on the center of a rectangle has
the following analog in Rosette:

\begin{verbatim}
(assert (equal? (rect-center r) (point 10 20)))
\end{verbatim}

As with Babelsberg, the expression is evaluated in the host language, and
\verb|(rect-center r)| will be partially evaluated away prior to reaching
constructs known to Rosette.  
%% The interpretation of the assertion can also
%% change depending on the binding of (for example) \verb|rect_center|.
However, the \verb|assert| statement just asserts the constraint at the
time it is evaluated, unlike \verb|always| in Babelsberg that re-asserts
the constraint at each time step.

Such persistent constraints declared using \verb|always|, as well as
the low-priority stay constraints, are not supported primitively in
Rosette, but it is straightforward to add them in a DSL on top of
Rosette, as described in the next section.

An important difference between Babelsberg and Rosette concerns the
handling of variables, state, and assignment for immutable types.  In
Babelsberg, for a variable that refers to an immutable object, a weak stay
constraint causes it to retain its value in the absence of an explicit
assignment or stronger constraints.  The value itself is however immutable.
Concrete values in Rosette are similar; but to make use of constraints,
programmers declare symbolic constants.  They can then assert constraints
on these symbolic constants and ask the solver for a satisfying solution.
Programmers can also make aliases for these symbolic constants, or pass
them to procedures and in those procedures put assertions on the symbolic
constants.  In contrast, in Babelsberg, aliasing doesn't apply to instances
of primitive types.  In Rosette, transparent immutable structures can hold
symbolic constants, and so the same considerations apply to them as to
symbolic constants themselves.  

Thus, in Babelsberg, types without object identity are immutable ---
constraints on variables and stay constraints are used to update the values
as time progresses.  In Rosette, symbolic constants, and immutable
transparent structures that contain them, form a kind of shell.  The
programmer can clear all the assertions on them, or even make assertions
that apply only within a particular call to \verb|solve|; but the structure
persists independent of these assertions.
%% A symbolic value (either of a
%% primitive type or a struct) may have a binding in the current solution, but
%% in another solution may have a different binding.
This concept of symbolic
values and different bindings does not exist in Babelsberg.  On the one
hand, the Rosette approach bypasses the issues with taming the power of the
constraint solver that arise in Babelsberg, since the solver just works on
the leaves (symbolic numbers, booleans, etc), and not on user-defined
structures.  On the other hand, the situation with ordinary objects in
Babelsberg, and mutable or opaque structs in Rosette, is much closer.  In
both systems, the solver can't make new objects or spontaneously change the
object to which a variable refers to satisfy constraints.


%%% LONG VERSION:

%% Object identity is handled in a similar manner in the two systems.  In both
%% systems, object identity is not relevant for primitive types, nor for
%% transparent immutable structures (in Rosette) or instances of value classes
%% (in Babelsberg).  Object identity is significant for opaque or mutable
%% structures (in Rosette) and for instances of mutable records or ordinary
%% classes (in Babelsberg).  

%% An important difference concerns how variables, state, and assignment are
%% handled, particularly for primitive types and instances of value
%% classes/transparent immutable structures.  In Babelsberg, programmers can
%% put constraints on the variable that must be satisfied for required
%% constraints, or that should be satisfied if possible for soft constraints.
%% Weak stay constraints cause variables to retain their values in the absence
%% of an explicit assignment or stronger constraints.  Concrete values in
%% Rosette are similar; but to make use of constraints, programmers declare
%% symbolic constants.  They can then assert constraints on these symbolic
%% constants and ask the solver for a satisfying solution.  Programmers can
%% also make aliases for these symbolic constants, or pass them to procedures
%% and in those procedures put assertions on the symbolic constants.  In
%% contrast, in Babelsberg, aliasing doesn't apply to instances of primitive
%% types.

%% Instances of user-defined value classes in Babelsberg are also
%% immutable and can be passed freely to procedures or methods without
%% the possibility of assigning to them or constraining them in the
%% procedure.  In Rosette, transparent immutable structures can hold
%% symbolic constants, and so the same considerations apply to them as to
%% symbolic constants themselves.  Thus, in Babelsberg, primitives and
%% other types without object identity are immutable --- constraints on
%% variables and stay constraints are used to update the values as time
%% progresses.  In Rosette, symbolic constants, and immutable transparent
%% structures that contain them, form a kind of shell.  The programmer
%% can clear all the assertions on them, or even make assertions that
%% apply only within a particular call to \verb|solve|; but the structure
%% persists independent of these assertions.  A symbolic value (either of
%% a primitive type or a struct) may have a binding in the current
%% solution, but in another solution may have a different binding.  This
%% concept of symbolic values and different bindings does not exist in
%% Babelsberg.  On the one hand, the Rosette approach bypasses the issues
%% with taming the power of the constraint solver that arise in
%% Babelsberg, since the solver just works on the leaves (symbolic
%% numbers, booleans, etc), and not on user-defined structures.  On the
%% other hand, the situation with ordinary objects in Babelsberg, and
%% mutable or opaque structs in Rosette, is much closer.  In both
%% systems, the solver can't make new objects or spontaneously change the
%% object to which a variable refers to satisfy constraints.

\subsection{Fran}
\label{sec:fran}

There has been substantial work on reactive programming languages and
frameworks --- see reference
\cite{bainomugisha-computing-surveys-2012} for a useful survey.  An
early and influential system is Fran by Conal Elliott and Paul Hudak,
and it has been an important inspiration for the work on adding
reactive constraints (Section \ref{sec:reactive}).  Fran provides an
integrated set of data types and functions in the Haskell language
designed for writing animations, in particular \emph{behaviors} and
\emph{events}.  As is implied by being written in Haskell, it is a
purely functional system, and graphical elements and other behaviors
are implicitly functions of time rather than having state that changes
with time.  In Fran, time is continuous, not discrete.  \emph{Interval
  analysis} \cite{snyder-siggraph-1992} is used to perform event
detection.

Typically, imperative reactive languages, for example Flapjax
\cite{meyerovich-oopsla-2009}, use behaviors and events as well.  In
all of these languages, functions or methods must then be
\emph{lifted} to operate on behaviors rather than ordinary values.

\section{The Core Wallingford System}
\label{sec:core-wallingford}

The core Wallingford system includes a small set of macros and methods
that extend Rosette for use in constructing interactive graphical
applications, as well as other applications involving objects with
state and persistent constraints.  The key features provided by the
core system are facilities for building objects with persistent
constraints, such as constrained geometric objects, GUI widgets with
constraints, or simulation elements; and for implementing frame axioms
(specifying that things remain the same unless there is some reason
they should change).  Soft constraints as well as hard (required) ones
are useful for both of these features.  In particular, the frame
axioms are implemented using low-priority ``stay'' constraints that
instruct the solver to leave the value of a variable as it was, unless
forced to change by some stronger constraint.

These features are provided by a class \verb|thing%| that includes the
following macros and methods.  (The Racket convention is that class
names end in {\tt \%}.)

\begin{description}

\item[{\tt always}] This macro takes a constraint and an optional
  priority, and adds that constraint to the set of constraints that will be
  enforced thereafter.  The priority defaults to {\tt required}.  For
  example, here is a constraint to keep a midpoint halfway between the two
  endpoints of a line:

\begin{verbatim}
(always (equal? midpoint 
  (point-times 0.5
    (point-plus (line-end1 line1)
                (line-end2 line1))))
\end{verbatim}

\item[{\tt always*}] This provides similar functionality to
  \verb|always|, except that the expression is re-evaluated at each
  step, so that the constraint applies to the new bindings of
  variables if they are re-assigned.

\item[{\tt stay}] This macro takes an expression and a priority (which
  defaults to ``lowest''), and adds a stay constraint to the system that
  the value of the expression stay the same each iteration.  For example,
  this constraint tells the system that one of the endpoints of a line
  should not move unless necessary, with low priority:

  \begin{verbatim}
  (stay (line-end1 line1) #:priority low)
  \end{verbatim}

\item[{\tt clear}]  This method cleans out all of the constraints
  and stays owned by a particular thing.

\item[{\tt solve}] This method is similar to Rosette's {\tt solve}
  macro, augmented with features to support Wallingford.  It finds and
  returns a solution to the explicit assertions plus all constraints
  declared using \verb|always| and \verb|stay|.  Stay constraints are
  considered relative to the current state of the thing.  The solution
  returned from {\tt solve} can then be used to evaluate an
  expression involving symbolic variables to find a concrete value.  The
  system uses iterative deepening to find a solution to the constraints
  that satisfies as many of the soft constraints as possible, respecting
  their relative priorities.  (Using the terminology of
  \cite{borning-lisp-symbolic-computation-1992}, specifically it finds a
  unsatisfied-count-better solution.)

\end{description}

%% The \verb|wallingford/applications| directory has a few examples of using
%% these functions and macros.  \verb|geothings.rkt| contains some definitions
%% of simple geometric objects with constraints. \verb|quadrilateral.rkt| uses
%% these to define an interactive demo that lets the user drag the endpoints
%% of lines that form a quadrilateral with an embedded parallelogram.
%% \verb|electrical-things.rkt| and \verb|electrical-things-dynamic.rkt| have
%% two styles of defining electrical circuit components; the unit tests in the
%% \verb|wallingford/tests| directory wire up some circuits using them.

\section{Adding Reactive Constraints}
\label{sec:reactive}

This section describes work in progress on adding temporal constraints and
reactivity to Wallingford.  Many of the examples and features are inspired
by Fran \cite{elliot-icfp-1997}, although, as a language with state, in
many ways it is not at all like Fran.  While the current design is being
developed using Rosette and the framework presented in the previous
section, the same ideas could be used in other constraint languages as
well.

Reactivity is encapsulated in a class \verb|reactive-thing%|, which is
a subclass of \verb|thing%|.  Each reactive thing has its own internal
time --- the plan is for time to be continuous, as in Fran, but in the
current implementation it is an integer (milliseconds since system
start time), due to Rosette limitations.\footnote{This limitation
  should be removed in the near future.  Z3 itself does provide real
  numbers, so this is a matter of making this functionality available
  in Rosette.  Note that these are mathematical reals, not floats.
  (Again, the goal at this point is getting a clean and expressive
  language, rather than performance --- there will be many
  opportunities for optimization later.)}

Reactive things have a default image, and we can set up a
\emph{viewer} that periodically samples the state of the reactive
thing and displays it.  This sampling can be done either on a pull or
a push basis --- the semantics are that of pulling (i.e., periodic
polling) --- push notification is an optimization that doesn't change
the semantics.  Reactive things understand the message
\verb|get-sampling|, which returns one of \verb|push|, \verb|pull|,
\verb|push-pull|, or \verb|none| (for static things).  The recommended
kind of sampling is determined automatically by analyzing the thing's
set of constraints.  Each reactive thing, as well as each viewer, has
its own thread, implemented using the Racket thread mechanism, and
communicate as needed.  For example, if pull sampling is being used,
the viewer will periodically poll the viewed thing to ask it to
advance time and then to show its current image; while if push is
being used, the thing will notify the viewer that something
potentially worth observing has occurred.

Constraints on reactive things can reference the thing's current time,
making them temporally dependent.  For example, we can create a reactive
thing that includes a blue circle \verb|c| as its image, and make the
circle smoothly change size as a function of time:

\begin{verbatim}
(always* (equal? (circle-radius c)
                 (+ 60 (* 50 (sin (seconds))))))
\end{verbatim}

In this case, there is a pure pull relationship between the viewer and
the viewed thing --- viewer polls the viewed thing as often as need be to
get a smooth animation.  (In the current implementation, this example runs
with some jerkiness, since the system is actually calling Z3, as an
separate process, to solve the constraint on each refresh.)  To do this,
the viewer sends the reactive thing a message \verb|advance-time| with a
proposed new time $t$ as an argument.  The reactive thing responds by
advancing its internal clock appropriately; and then its image can be
retrieved and shown using the new time $t$\@.  Semantically, the viewed
thing takes on all of the times between its current time and $t$, but since
there are no detectable effects of taking on all the intermediate states,
the thing's internal time can simply jump to $t$.

Conditionally asserted constraints are supported using \verb|when| and
\verb|while| macros.

\begin{description}
    
\item[when] takes a boolean-valued expression and asserts a set of
  constraints at the instant that event occurs.

\item[while] similarly takes a boolean-valued expression, and asserts a set
  of constraints that should hold as long as the condition has the value
  true.

%% \item[integral] \verb|(integral f t0)| is the definite integral
%%   $\int_{t_0}^t f \, dt$, where $f$ is some expression, $t_0$ is a start
%%   time and $t$ is the current time.

%% \item[detect] takes a boolean-valued expression and returns an event
%%   that is true at the instant the value of the expression changes to
%%   true.

\end{description}

Both \verb|when| and \verb|while| have a boolean-valued expression as their
conditions, but for \verb|when| the expression is restricted to denoting an
\emph{event} that is true just for an instant.\footnote{We can formalize
  this restriction as follows.  For any time $t$ such that the condition
  $c$ of the {\tt when} has the value true, there exists some $\delta>0$
  such that $c$ is false for all other times in the interval
  $[t-\delta,t+\delta]$.}  The default duration for a \verb|while| or
\verb|when| macro is \verb|always|: it should hold from that point onward.
These macros also allow for an \verb|until| clause with an event-valued
expression, which specifies that the \verb|while| or \verb|when| should be
deactivated if the event occurs.

For example, we can add a \verb|when| temporal constraint to our 
circle to make it flip between blue and red every 5 seconds:

\begin{verbatim}
(define (flip c)
  (if (equal? c (color "blue"))
      (color "red")
      (color "blue")))

(when (zero? (remainder (milliseconds) 5000))
  ((equal? (colored-circle-color c) 
    (flip (previous (colored-circle-color c))))))
\end{verbatim}

If a reactive thing includes \verb|when| or \verb|while| temporal
constraints, handling the \verb|advance-time| message is more complex,
since a \verb|when| condition might be true at some instant between
its current time and the new proposed time, or a \verb|while|
condition might change from false to true or vice versa.  Even in the
current implementation, in which time is represented as an integer, it
would be quite inefficient to methodically advance time 1 millisecond
per tick, checking each tick for \verb|when| conditions that have
become true or \verb|while| conditions that change value; and
impossible when the implementation is rewritten to use real-valued
time.  Instead, the system uses its constraint solving capabilities to
find the next time to which to advance.  Let the reactive thing's
current time be $r$, and the new proposed time be $t$\@.  The system
then solves for a time $s$ such that $r<s<t$, all of the required
constraints hold, and at least one condition in a \verb|when| holds or
a \verb|while| condition changes its value.  If no such time exists,
the reactive thing can safely advance its time to $t$\@.  But if such
a time $s$ does exist, the system uses iterative deepening to find the
minimum such time $s_{\min}$, and the thing advances its clock to
$s_{\min}$.  It then uses the Wallingford version of \verb|solve| to
solve the constraints at $s_{\min}$ (including the soft constraints),
asserts the constraints in the bodies of all \verb|when| or
\verb|while| statements whose condition holds at $s_{\min}$ (for a
\verb|when|, just at that instant, and for a \verb|while|, as long as
the condition holds --- like a bounded version of \verb|always|), and
finally calls \verb|solve| again.  Finally, if there were such a time
$s_{\min}<t$, the reactive thing calls \verb|advance-time| again to
attempt to advance its time to $t$\@.  (It might need to advance to
multiple intermediate times before eventually getting to $t$.)

The function \verb|previous| used above denotes the value of the circle's
color at some time $p$ less than the current time and greater than any
other time that the thing's state has been accessed.  (Thus,
\verb|previous| can only be used in a \verb|when| constraint --- it doesn't
make sense in a \verb|while|.)

Incidentally, the condition in the above \verb|when| takes advantage of
milliseconds being an integer --- when this is converted to use reals, the
condition would need to also include a test
\verb|(integer (milliseconds))|.

External events can also be accommodated in this framework.  For example,
the function \verb|left-button-pressed| returns true just at those
instants that the button is pressed.  (Note that this is not the same as
\verb|left-button-down|, which returns true throughout the interval
the button is down --- the ``pressed'' function returns true just at one
instant.)  We can then use such functions in \verb|when| conditions, for
example to flip the color when the button is pressed:

\begin{verbatim}
(when (left-button-pressed)
  ((equal? (colored-circle-color c) 
    (flip (previous (colored-circle-color c))))))
\end{verbatim}

If the reactive thing is monitoring the button state, then it
maintains a list of times that the button is pressed; and when
advancing time, we need to advance to the first time such that a
condition on a \verb|when| is true or the condition on a \verb|while|
changes value, including ones involving these external events.  For
example, suppose that a reactive thing \verb|r| includes the above
constraint, and there is a button press at time 2000 (milliseconds).
If \verb|r| gets an \verb|advance-time| message asking it to advance
to 1000, it can simply do so.

\newpage

On the other hand, if it is asked to
advance time to 3000, then that method will find that the \verb|when|
condition on the flip constraint is satisfied at some earlier time, so
it will just advance to 2000 instead.  The constraints in the body of
the \verb|when| will be satisfied, and it will try again to advance to
3000, and assuming there is no intervening button press, it can do so.

As noted in Section \ref{sec:fran}, other reactive languages, both
purely functional ones such as Fran, and imperative ones such as
Flapjax, typically use separate types for behaviors and events,
requiring that ordinary functions or methods be lifted to operate on
them.  In contrast, in Wallingford, in place of a behavior to
represent for example a button down event, \verb|left-button-pressed|
is simply an ordinary function that returns true or false: the
complexity is removed here, and instead occurs in the implementation
of \verb|advance-time|, which is provided by the underlying system and
hence need only be written once by the Wallingford implementor.  Thus,
the hope is that the programmer will be presented with a much simpler
model.

\section{Future Work}

This short paper is a snapshot of work-in-progress.  As of this
writing, the class \verb|reactive-thing%|, along with \verb|when|
temporal constraints, the code to automatically determine the sampling
scheme, and the class \verb|viewer%|, plus unit tests and some example
programs, is implemented and tested.  Still remaining is implementing
\verb|while|.  The next steps are thus to complete the prototype
implementation of the reactive programming constructs and to test them
on more complex interactive applications.  Two additional temporal
constraints are also planned: \verb|(integral f)|, which denotes the
definite integral $\int_{t_0}^{t_1} f \, dt$ for an expression $f$, where
$t_0$ is the time at which the integral constraint is asserted and $t_1$
is the current time (modeled after an analogous construct in Fran);
and \verb|(detect expr)|, which takes a boolean-valued expression and
returns an event that is true at each instant the value of \verb|expr|
changes to true.

Following that, getting reasonable performance through compilation and
other techniques will be essential.  Other future work will include
developing a more formal semantics for the temporal constraints, using
Rosette's capabilities to support inferring constraints from examples, and
perhaps working out how Wallingford could function in a distributed system
with multiple clocks.

\acks

Many thanks to colleagues in UW Computer Science \& Engineering and
the Communications Design Group, as well as to the CROW referees, for
useful feedback on this work.  This research was funded in part by SAP
and Viewpoints Research Institute.

\newpage

\renewcommand{\bibfont}{\normalsize}
\bibliographystyle{abbrvnat}

% \bibliography{constraints}

\begin{thebibliography}{13}
\providecommand{\natexlab}[1]{#1}
\providecommand{\url}[1]{\texttt{#1}}
\expandafter\ifx\csname urlstyle\endcsname\relax
  \providecommand{\doi}[1]{doi: #1}\else
  \providecommand{\doi}{doi: \begingroup \urlstyle{rm}\Url}\fi

\bibitem[Bainomugisha et~al.(2012)Bainomugisha, Carreton, Cutsem, Mostinckx,
  and Meuter]{bainomugisha-computing-surveys-2012}
E.~Bainomugisha, A.~L. Carreton, T.~V. Cutsem, S.~Mostinckx, and W.~D. Meuter.
\newblock A survey on reactive programming.
\newblock \emph{ACM Computing Surveys}, 2012.

\bibitem[Borning et~al.(1992)Borning, Freeman-Benson, and
  Wilson]{borning-lisp-symbolic-computation-1992}
A.~Borning, B.~Freeman-Benson, and M.~Wilson.
\newblock Constraint hierarchies.
\newblock \emph{Lisp and Symbolic Computation}, 5\penalty0 (3):\penalty0
  223--270, Sept. 1992.

\bibitem[De~Moura and Bj{\o}rner(2008)]{demoura-z3-2008}
L.~De~Moura and N.~Bj{\o}rner.
\newblock Z3: An efficient {SMT} solver.
\newblock In \emph{Tools and Algorithms for the Construction and Analysis of
  Systems}, pages 337--340. Springer, 2008.

\bibitem[Elliott and Hudak(1997)]{elliot-icfp-1997}
C.~Elliott and P.~Hudak.
\newblock Functional reactive animation.
\newblock In \emph{International Conference on Functional Programming}, 1997.
\newblock URL \url{http://conal.net/papers/icfp97/}.

\bibitem[Felgentreff et~al.(2014)Felgentreff, Borning, and
  Hirschfeld]{FelgentreffJOT14}
T.~Felgentreff, A.~Borning, and R.~Hirschfeld.
\newblock Specifying and solving constraints on object behavior.
\newblock \emph{Journal of Object Technology}, 13\penalty0 (4):\penalty0 1--38,
  2014.

\bibitem[Felgentreff et~al.(2015{\natexlab{a}})Felgentreff, Millstein, and
  Borning]{felgentreff-step-by-step-tr}
T.~Felgentreff, T.~Millstein, and A.~Borning.
\newblock Developing a formal semantics for {Babelsberg}: A step-by-step
  approach.
\newblock Technical Report 2014-002b, Viewpoints Research Institute,
  2015{\natexlab{a}}.
\newblock Available at \url{http://www.vpri.org/pdf/tr2014002_babelsberg.pdf}.

\newpage

\bibitem[Felgentreff et~al.(2015{\natexlab{b}})Felgentreff, Millstein, Borning,
  and Hirschfeld]{felgentreff-oopsla-2015}
T.~Felgentreff, T.~Millstein, A.~Borning, and R.~Hirschfeld.
\newblock Checks and balances --- constraint solving without surprises in
  object-constraint programming languages.
\newblock In \emph{OOPSLA}, Oct. 2015{\natexlab{b}}.

\bibitem[Jaffar and Lassez(1987)]{jaffar-popl-1987}
J.~Jaffar and J.-L. Lassez.
\newblock Constraint logic programming.
\newblock In \emph{Proceedings of the Fourteenth ACM Principles of Programming
  Languages Conference}, Munich, Jan. 1987.

\bibitem[Meyerovich et~al.(2009)Meyerovich, Guha, Baskin, Cooper, Greenberg,
  Bromfield, and Krishnamurthi]{meyerovich-oopsla-2009}
L.~A. Meyerovich, A.~Guha, J.~Baskin, G.~H. Cooper, M.~Greenberg, A.~Bromfield,
  and S.~Krishnamurthi.
\newblock Flapjax: A programming language for {Ajax} applications.
\newblock In \emph{OOPSLA}, 2009.

\bibitem[Racket(2016)]{racket}
Racket.
\newblock The {Racket} programming language, 2016.
\newblock \url{http://racket-lang.org}.

\bibitem[Snyder(1992)]{snyder-siggraph-1992}
J.~M. Snyder.
\newblock Interval analysis for computer graphics.
\newblock \emph{ACM SIGGRAPH Computer Graphics}, 26\penalty0 (2):\penalty0
  121--130, 1992.

\bibitem[Torlak and Bodik(2013)]{torlak-onward-2013}
E.~Torlak and R.~Bodik.
\newblock Growing solver-aided languages with rosette.
\newblock In \emph{Onward!} ACM, 2013.

\bibitem[Torlak and Bodik(2014)]{torlak-pldi-2014}
E.~Torlak and R.~Bodik.
\newblock A lightweight symbolic virtual machine for solver-aided host
  languages.
\newblock In \emph{PLDI}. ACM, 2014.

\end{thebibliography}

\end{document}

%%  LocalWords:  rightsretained Modularity '16 ' laga Wallingford Borning
%%  LocalWords:  DSLs ccs2012 desc Babelsberg Javascript Smalltalk OO Z3
%%  LocalWords:  Babelsberg's DSL struct booleans structs Conal Hudak UW
%%  LocalWords:  subdirectory reactivity boolean Javascript
